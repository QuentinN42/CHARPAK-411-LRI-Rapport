

\newcommand{\none}{\textit{}\\}

\newcommand{\mathematica}{\href{https://www.wolfram.com/mathematica/}{\textsc{Mathematica }}}
\newcommand{\lamap}{\href{https://www.fondation-lamap.org/}{\textit{"la main a la pâte" }}}
\newcommand{\ivgc}{\href{http://www.villebon-charpak.fr/}{\textit{Institut Villebon Georges Charpak }}}


% config
\pagestyle{plain}
\floatplacement{figure}{H}

\captionsetup{justification=centering}


\def \grand {200pt}
\def \moyen {150pt}
\def \petit	{100pt}
\def \mini	{ 50pt}


\def \vo    {v_{0}}
\def \reau  {\rho_{eau}}
\def \ux    {\vec{u_{x}}}


\newcommand{\sujet}{Algorithmes de minimisation du regret et integrales de choquet}
\newcommand{\johanne}{Johanne \textsc{Cohen}}
\newcommand{\lri}{\textsc{LRI}}
\newcommand{\Lri}{laboratoire de recherche en informatique}
\newcommand{\galac}{\textsc{GALaC}}
\newcommand{\sgd}{\textsc{sgd}}


\newcommand{\bbox}[2]
{
\begin{tcolorbox}[
    colback=blue!1,
    colframe=blue!40!black,
    title=
    \textbf{#1}
    ]
	#2
\end{tcolorbox}
}


\newcommand{\rbox}[2]
{
\begin{tcolorbox}[
    colback=red!1,
    colframe=red!40!black,
    title=
    \textbf{#1}
    ]
	#2
\end{tcolorbox}
}


\newcommand{\exemle}[1]
{
	\bbox{Exemple(s) :}{#1}
}


\newcommand{\img}[4]{ %taille,lien,description,label
\begin{figure}
	\begin{center}
		\includegraphics[height=#1]{#2}
	\end{center}
	\caption{#3}
	\label{#4}
\end{figure}
}

