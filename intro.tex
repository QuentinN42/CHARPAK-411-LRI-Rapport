
\section{Objectif du stage}
\label{sec:obj_rapide}

Mon stage c'est déroulé au \Lri, le sujet de mon stage était le suivant : \textit{\sujet}.
Je n'avais jamais fait d'informatique théorique (exepté IA)
et n'ayant jamais vus la théorie de la mesure,
il m'était compliqué de comprendre la notion d'intégrale de choquet (découlant des intégrale de Lebesgue).
Ainsi, durant ce rapport, la partie mathématique théorique ne sera pas vue en detail.

\section{Présentation générale du \lri}
\label{sec:pglri}

Le Laboratoire de Recherche en Informatique (\lri) est une unité de recherche de l'Université Paris-Sud et du \textsc{CNRS}.
Créé il y a plus de 35 ans, le laboratoire est localisé sur le plateau du Moulon depuis début 2013.
Il accueille plus de 250 personnes dont environ un tier de doctorants.


Organisés en neuf équipes, les recherche du laboratoire incluent à la fois des aspects théoriques et appliqués (ex: algorithmique, réseaux et bases de données, graphes, bioinformatique, interaction homme-machine, ect).
De part cette diversité, le laboratoire favorise les recherches aux frontières de différents domaines, là où le potentiel d'innovation est le plus grand.
En plus de son activité de publication (2000 publications entre début 2008 et mi 2013), le LRI dévelope de nombreux logiciels.


Pour plus d'informations, je vous invite à aller regarder sur leur site ou une présentation détaillée est continuellement tenue a jour\cite{LRI}.


\section{Unitée \galac}
\label{sec:galac}

Durant mon stage, j'ai travaillé sous la direction de \johanne\ qui est la responsable qe l'équipe \galac. \\
L'équipe \galac\ est une équipe du \lri\ qui travaillent sur des thématiques théoriques comme
l'algorithmique, la théorie des graphes ou les systèmes en réseaux.
