
Mon stage c'est déroulé au \Lri, son sujet était le suivant : \textit{\sujet}.
Je n'avais jamais fait d'informatique fondamentale (exepté IA)
et n'ayant jamais vus la théorie de la mesure,
il m'était d'aprhender en totalité la notion d'intégrale de Choquet (découlant des intégrale de Lebesgue).
C'est pourquoi, durant ce rapport, la partie mathématique théorique ne sera pas vue en detail.\\
\textcolor{red}{Organisation du document}

\section{Présentation générale du \lri}
\label{sec:pglri}

Le Laboratoire de Recherche en Informatique (\lri) est une unité de recherche de l'Université
Paris-Sud et du \textsc{CNRS}.
Créé il y a plus de 40 ans, le laboratoire est localisé sur le plateau du Moulon depuis début 2013.
Il accueille plus de 250 personnes dont environ un tier de doctorants.


Organisés en neuf équipes, les recherches du laboratoire incluent à la fois des aspects théoriques et appliqués
(ex: algorithmique, réseaux et bases de données, graphes, bioinformatique, interaction homme-machine, ect).
De part cette diversité, le laboratoire favorise les recherches aux frontières de différents domaines,
là où le potentiel d'innovation est le plus grand.
En plus de son activité de publication (2000 publications entre début 2008 et mi 2013),
le LRI dévelope de nombreux logiciels.


Pour plus d'informations, je vous invite à aller regarder sur leur site
ou une présentation détaillée est continuellement tenue à jour\cite{LRI}.


\section{Équipe \galac}
\label{sec:galac}

Durant mon stage, j'ai travaillé sous la direction de \johanne\ qui est la responsable de l'équipe \galac. \\
L'équipe \galac\ est une équipe du \lri\ composée d'environ 20 chercheurs issus du \textsc{cnrs}, de l'Université
Paris-Sud et de Centrale Supelec qui travaillent sur des thématiques théoriques comme
l'algorithmique, la théorie des graphes ou les systèmes en réseaux.
Mon travail s'intègre dans cette équipe puisqu'il réside dans l'étude du fonctionnement d'un aglorithme.
