Une base de donnée disponible en ligne sur \textsc{Kaggle} à aussi été étudiée\cite{kaggle}.
La base de donnée traite des prix de maisons vendues entre $2014$ et $2015$
dans le King Country, \textsc{wa}, \textsc{usa}.\\


Pour rechercher les fonctions d'utilitées, une liste de fonctions possible à été dressée:
\begin{itemize}
    \item Polynomes de degrés allant de $1$ à $3$
    \item Exponentiel
    \item Logarithme
    \item Hyperbolique
    \item Racine
    \item Sigmoide
    \item Gaussienne
\end{itemize}
Une fois toutes ces fonctions codées, chaque plage de données à essayé d'être regressée grâce à celles ci.
La fonction obtenant le meilleur $R^2$ à été conservé si celui ci était superieur à $0.3$ afin de retirer
les données n'agissant pas sur le prix.\\


En utilisant les fonctions d'utilitées trouvées precedement, une réseau de choquet à été créé puis entrainé
sur cette base de données.
L'entrainement est cependant long, environ $8$h sur cluster de calcul du \lri.
Ce temps à pu être réduit grace à de l'optimisation et de la programmation parallèle.
