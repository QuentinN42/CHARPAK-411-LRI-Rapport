

\paragraph{Données artificielles (\ref{subsec:artif}) :\\}
Durant cette étude, l'efficacité des réseaux de neurones a pu être observée.
Il n'est plus à démontrer la puissance de généralisation ainsi que la résistance aux perturbations de cette technique.
De plus, l'influence du nombre d'apprentissages ainsi que du nombre de données disponibles sur la fiabilité des résultats ont pu être étudié.
Ces résultats démontrent qu'un réseau peut apprendre efficacement même si peu de données lui sont fournies.



\paragraph{Données réelles (\ref{subsec:real}) :\\}
Les résultats sur les données réelles ne sont pourtant pas à la hauteur
de ce que le modèle théorique (\ref{subsec:réseau-de-neurones})
et que les tests réels (\ref{subsec:artif}) laissait présager.
En effet, sans traitement particulier, l'apprentissage est totalement chaotique.
Les fonctions d'utilités permettent d'améliorer l'apprentissage mais
on observe grâce au réseau de Choquet que les plages de données sélectionnées
sont des facteurs de confusions impliquant des problèmes d'overfitting.


\paragraph{Dévelopement personel :\\}
Ce stage m'a appris de nombreuses choses sur le monde de la recherche tel
que la manipulation de modèles mathématiques théoriques et leurs applications
en informatique fondamentale.


J'ai aussi pu implémenter mon premier réseau de neurones.
J'avais déjà des connaissances théoriques mais je
n'avais jamais eu l'occasion de manipuler ce genre d'objets.


L'utilisation scientifique de python a aussi été une part très importante de mon stage.
Des problématiques qui n'avaient jamais été abordées en cours tel que l'optimisation se sont posées.
Les échanges avec les autres stagiaires/thésards ont été très instructifs
(découverte de nouvelles librairies, bonnes méthodes à adopter\ldots).
J'ai aussi eut l'opportunité d'utiliser un cluster de calcul du \lri.


Enfin, durant ce stage, j'ai pu échanger avec des universitaires et des enseignants chercheurs
sur ma poursuite d'études.
J'ai par exemple pu découvrir les doctorats \textsc{CIFRE}\cite{cifre}:
Ce sont des doctorats en partenariat avec une entreprise c'est donc l'une des meilleures
orientations pour faire de la recherche & développement au sein d'une entreprise.


\paragraph{Perspectives :\\}
Il serait tout d'abord intéressant de savoir si ce modèle overfit.
Pour cela il faudrait tester celui-ci sur une autre base de données.
Les résultats permettraient de conclure sur l'efficacité de ce réseau.


Pour pouvoir tirer de plus amples résultats il serait bien de tester la variation de
l'apprentissage en fonction du nombre d'entrées du réseau utiles et inutiles.
Par exemple un réseau faisant un produit scalaire entre son vecteur poids et son vecteur entrée,
il permettra de définir une approximation de la taille de la base de données nécessaire à approximer une fonction.
L'efficacité d'un réseau faisant la somme de ses deux premières entrées en fonction de la profondeur de l'apprentissage
peut aussi être utile pour tester la résistance du réseau aux facteurs de confusion.
