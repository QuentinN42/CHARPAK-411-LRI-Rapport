
Il faut cependant nuancer les résultats obtenus dans la partie\ \ref{subsec:real}.
En effet, il se peut que cette base de données soit biaisée,
il n'est donc pas possible de tirer des conclusions générales à partir de ce cas particulier.
De plus, toutes les fonctions d'utilités n'ont pas été obtenues.
Il est possible que certaines fonctions qui ne sont pas présentes dans\ \ref{sec:données-réelles}\
soient d'excellentes fonctions d'utilités.


\paragraph{Dévelopement personel :\\}
Ce stage m'a appris de nombreuses choses sur le monde de la recherche tel
que la manipulation de modèles mathématiques théoriques et leurs applications
en informatique fondamentale.\\


J'ai aussi pu implémenter mon premier réseau de neurones.
J'avais déjà des connaissances théoriques mais je
n'avais jamais eu l'occasion de manipuler ce genre d'objets.\\


L'utilisation scientifique de python a aussi été une part très importante de mon stage.
Des problématiques qui n'avaient jamais été abordées en cours tel que l'optimisation ce sont possées.
Les échanges avec les autres stagiaires/thésards ont été très instructifs
(découverte de nouvelles librairies, bonnes méthodes à adopter\ldots).
J'ai aussi eut l'opportunité d'utiliser un cluster de calcul du \lri.\\


Enfin, durant ce stage, j'ai pu échanger avec des universitaires et des enseignants chercheurs
sur ma poursuite d'études.
J'ai par exemple pu découvrir les doctorats \textsc{CIFRE}\cite{cifre}:
Ce sont des doctorats en partenariat avec une entreprise c'est donc l'une des meilleures
orientations pour faire de la recherche & développement au sein d'une entreprise.
