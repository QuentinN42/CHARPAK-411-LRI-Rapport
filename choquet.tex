
L'integrale de choquet est une intégrale découlant de la théorie de la mesure que je n'ai pas vus,
de plus, ici on ne parlera que du modele discret.\\


Pour l'expliquer simplement, prenons un exemple :
Suposons que l'on veuille conseiller des personnes sur l'achat d'un oridnateur.
Ces personnes de conaissent absolument rien en informatique.
La seul chose qu'ils veullent c'est une note, plus elle est élevée, plus l'ordinateur est performant.
Essayons de faire un algorithme assez simple pour résoudre ce problemme :
Chaque composant se voit atribué une note (en fonction de la puissance, la qualité de fabrication\ldots),
on appelle cette note \textit{l'utilité} du composant.
Une fois toutes les utilitées étudiées, on donne un poid à chacuns des composant.
Enfin on fait la somme de utilité fois poid pour tout les composant.\\
Ainsi, si un ordinateur a de meilleurs composant, plus de puissance, ect, sa note sera superieure.\\


Ce modèle parait raisonable dans la plupart des cas mais il est extraimement mauvais dans les cas extremes :
Suposons qu'un constructeur peu scrupuleux propose un ordinateur assez étrange :
Le processeur le moins cher du marché (assez mauvais) mais énormément de \textsc{ram}, par exemple $128$Go.
Votre classement le placera forcément en haut de la liste, même si vous en conviendrez,
cet ordinateur est assez inutile\ldots\\


Il faudrait donc trouver un autre système modélisant les interactions entre les composants.
Ce modèle est exactement celui sur lequel j'ai travaillé durant mon stage: \textit{les intégrales de choquet}.
En plus de donner un poid aux utilité,
un poid est donné aux interactions deux a deux des utilitées par le biais des fonctions min et max.\\


Voici donc la fonction que l'on vas vouloir aproximer :
\begin{equation}
    \label{choquet}
    C(X)  =
    \sum_{i=1}^{n}
        w_i \times x_i +
    \sum_{i=1}^{n}\sum_{j=i+1}^{n}
    \Big(
        w_{M\,ij} \times \max(x_i,x_j) + w_{m\,ij} \times \min(x_i,x_j)
    \Big)
\end{equation}
Avec $X$ le vecteur des utilitées et $W$, $W_m$ et $W_M$ les vecteurs des poids.