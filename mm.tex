Ce travail s'est déroulé en deux principales phases :
Une étape de dévelopement durant laquelle des données et la fonction à apprendre ont été générées.
Et une étape plus appliquée durant laquelle une base de données réelle a été étudiée.\\


Afin d'implémenter informatiquement les concepts théoriques vu précédemment,
il a été choisi d'utiliser la librairie Keras\cite{keras}, une réference en python pour faire du machine learning.
La partie sur les intégrales de Choquet a été entièrement recodée en python
avec la librairie numpy\cite{numpy} pour optimiser le temps de calcul.
L'ensembe du code est disponible gratuitement et sous licence libre sur github\cite{repoStage}.

\subsection{Keras}\label{subsec:keras}
La librairie Keras est une interface Python/TensorFlow\cite{tf} permetant de travailler avec des réseaux de neurones.
Ici, on ne s'attardera que sur les fonctionnalitées principales,
a savoir la création d'un réseau simple,
la regression par \sgd\ et l'évaluation des performances.


Voici un exemple de réseau de neurone assez simple
qui vas essayer de deviner l'application linéaire suivante : $f(X) = 0.2x_1 + 0.8x_2$.

\begin{figure}[H]
    \center
    \includegraphics[height=\petit]{pict/net2.png}
	\caption{Réseau simple}
	\label{fig:net2}
\end{figure}
\vspace{-12pt}


Pour créer ce réseau et lui faire apprendre la fonction precedement citée,
le code suivant est nescessaire :
\lstinputlisting[language=Python]{code/reseau1.py}


En executant ce code, on obtient:
\begin{lstlisting}
w0 : 0.19988934695720673
w1 : 0.8001111745834351
\end{lstlisting}
On peut donc bien voir que le réseau de neurones fonctionne et
réussit à apprendre des fonctions avec plusieurs paramètres.
Il est cependant assez embetant de toujours devoir faire appel a toutes ces fonctions.
Une librairie à alors été codée afin de simplifier son utilisation.
Elle pourra être appelée avec de nombreux paramtres qui seront abordés dans les parties suivantes.\\


\newpage
\subsection{Implementation d'un réseau de Choquet}\label{subsec:implementation-Choquet}
Nous appellerons ici \emph{réseau de choquet} un réseau de neurones ayant une architecture
adaptée a la regression d'une intégrale de choquet.
Comme vus precedement (\ref{subsec:intégrales-de-choquet}),
une fonction de choquet a une architecture complexe.
Voici l'intégrale de choquet entierement dévelopée pour un vecteur d'entrée taille $3$:
\begin{align*}
    C&\ =
    \color{red}
    w_1 \times x_1 + w_2 \times x_2 + w_3 \times x_3 \\
    \color{black}
    & +
    \color{blue}
    w_{m1} \times \min(x_1, x_2) + w_{m2} \times \min(x_1, x_3) + w_{m3} \times \min(x_2, x_3) \\
    \color{black}
    & +
    \color{green}
    w_{M1} \times \max(x_1, x_2) + w_{M2} \times \max(x_1, x_3) + w_{M3} \times \max(x_2, x_3)
\end{align*}
Voici l'architecture d'un réseau de choquet entierement générée par un réseau de neurones :

\begin{figure}[H]
    \center
    \includegraphics[height=\moyen]{pict/chnet1}
	\caption{Architecture d'un réseau de choquet}
	\label{fig:chnet1}
    \vspace{-10pt}
    \begin{center}
        \tiny
        \textit{
        En bleu, des neurone appliquant pour fonction principale $\min(X)$. \\
        En vert, des neurone appliquant pour fonction principale $\max(X)$.
        }
    \end{center}
\end{figure}
\vspace{-12pt}
On peut voir que trois problemes non triviaux se posent (\textit{cf.}\ \ref{subsec:réseau-de-neurones}) :
\begin{itemize}
    \item Les neurones collorées n'appliquent pas une simple application linéaire.
    \item Les neurones ne sont pas reliés en mode Dense mais en convolution.
    \item Certains neurones passent des informations en sautant une couche de neurones.
\end{itemize}


\paragraph{Une autre piste à alors été envisagée :}
Créer un réseau simple comme dans la figure\ \ref{fig:net3}.
Dans ce réseau, aucuns neurone n'a de fonction complexe :
ceux de gauche sont les entrées et
celui de droite fait le produit scalaire avec un vecteur poids.
\begin{figure}[H]
    \center
    \includegraphics[height=\petit]{pict/net3.png}
	\caption{Réseau alternatif}
	\label{fig:net3}
\end{figure}
\vspace{-12pt}
Ce réseau est bien plus simple à générer : c'est un réseau \emph{Dense} de taille $9$ en entrée.
Et plus généralement $n^2$ pour un vecteur de taille $n$.


\rbox{Démonstation :}
{
Prenons un vecteur de taille $n$,
Le but est d'énumerer le nombre de neurones utiles a l'équation\ (\ref{eq:choquet}).
Le resultat est le suivant :
\begin{equation}
    n + \dbinom{n}{2} + \dbinom{n}{2} = n + 2 \times \frac{n(n-1)}{2} = n^2
\end{equation}
}

Il faut alors créer un vecteur d'entrée à partir une base de donnée d'aprentissage.
Cela se fait simplement en concaténant le vecteur $X$ avec les elements pris deux à deux passés
dans les fonctions min et max.\\


Lors de la descente de gradient, le réseau traite les poids indiférement,
pour les récupérer les poids de l'équation\ (\ref{eq:choquet})
il sufit de prendre les $n$ premiers pour $W$,
les $\frac{n(n-1)}{2}$ suivant pour $W_m$
et les $\frac{n(n-1)}{2}$ derniers pour $W_M$.


\newpage
\subsection{Données réelles}\label{subsec:données-réelles}
Une base de donnée disponible en ligne sur \textsc{Kaggle} à aussi été étudiée\cite{kaggle}.
La base de donnée traite des prix de maisons vendues entre $2014$ et $2015$
dans le King Country, \textsc{wa}, \textsc{usa}.\\


Pour rechercher les fonctions d'utilitées, une liste de fonctions possible à été dressée:
\begin{itemize}
    \item Polynomes de degrés allant de $1$ à $3$
    \item Exponentiel
    \item Logarithme
    \item Hyperbolique
    \item Racine
    \item Sigmoide
    \item Gaussienne
\end{itemize}
Une fois toutes ces fonctions codées, chaque plage de données à essayé d'être regressée grâce à celles ci.
La fonction obtenant le meilleur $R^2$ à été conservé si celui ci était superieur à $0.3$ afin de retirer
les données n'agissant pas sur le prix.\\


En utilisant les fonctions d'utilitées trouvées precedement, une réseau de choquet à été créé puis entrainé
sur cette base de données.


